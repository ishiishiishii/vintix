\chapter{結果} \label{results}

本章では,前章で述べた実験設定に基づいて実施した評価実験の結果を示す.
まず,4種類すべてのロボットで訓練した Vintix モデルのマルチタスク性能を報告する.
次に,ゼロショット汎化性能の評価結果を示す.
最後に,エンコーダ・デコーダを分離した場合の性能について述べる.

\section{マルチタスク性能の評価}

\subsection{全ロボットで訓練したモデルの性能}

4種類すべてのロボット(Go2,Go1,A1,Mini Cheetah)のデータを用いて訓練した
Vintix モデルの性能を評価した.
各ロボットにおける1ステップあたりの平均報酬の推移を図\ref{fig:multitask_training}に示す.

% 注:グラフ挿入箇所
% \begin{figure}[t]
% \centering
% \includegraphics[width=0.8\textwidth]{figures/multitask_training_curves.png}
% \caption{全ロボットで訓練したVintixモデルの各ロボットにおける1ステップあたりの平均報酬の推移}
% \label{fig:multitask_training}
% \end{figure}

訓練済みモデルとPPO専門家ポリシーの性能比較を表\ref{tab:multitask_comparison}に示す.

\begin{table}[t]
\centering
\caption{全ロボットで訓練したVintixモデルとPPO専門家ポリシーの性能比較(1ステップあたりの平均報酬)}
\label{tab:multitask_comparison}
\begin{tabular}{lcc}
\toprule
\textbf{ロボット} &
\textbf{Vintix} &
\textbf{PPO専門家} \\
\midrule
Go2 & 0.017424 & 0.021352 \\
Go1 & 0.021587 & 0.022521 \\
A1 & 0.019938 & 0.020857 \\
Mini Cheetah & 0.019235 & 0.021014 \\
\bottomrule
\end{tabular}
\end{table}

実験の結果,Vintix モデルは4種類すべてのロボットで歩行を実現することができた.
しかし,各ロボット専用に訓練した PPO 専門家ポリシーと比較すると,
1ステップあたりの平均報酬は専門家ポリシーよりも低い値を示した.
これは,単一モデルで複数ロボットの制御を学習する際のトレードオフとして,
各ロボットに対する最適化が専門家ポリシーほど徹底されていないことを示唆している.

\section{ゼロショット汎化性能の評価}

\subsection{1種類ずつ除外した場合の性能}

各ロボットを1つずつ訓練データから除外し,
残りの3種類のロボットで訓練したモデルのゼロショット性能を評価した.
これにより,各ロボットのデータが汎用性向上にどの程度寄与するかを分析した.

各除外条件におけるゼロショット性能の推移を図\ref{fig:zeroshot_training}に示す.

% 注:グラフ挿入箇所
% \begin{figure}[t]
% \centering
% \includegraphics[width=0.8\textwidth]{figures/zeroshot_training_curves.png}
% \caption{各ロボットを除外した場合のゼロショット性能の推移}
% \label{fig:zeroshot_training}
% \end{figure}

ゼロショット性能の評価結果を表\ref{tab:zeroshot_results}に示す.

\begin{table}[t]
\centering
\caption{各ロボットを除外した場合のゼロショット性能(1ステップあたりの平均報酬)}
\label{tab:zeroshot_results}
\begin{tabular}{lcccc}
\toprule
\textbf{訓練データ} &
\textbf{Go2} &
\textbf{Go1} &
\textbf{A1} &
\textbf{Mini Cheetah} \\
\midrule
Go1, A1, Mini Cheetah & 0.010508 & 0.019940 & 0.020282 & 0.018648 \\
Go2, A1, Mini Cheetah & 0.020887 & 0.021539 & 0.019970 & 0.018276 \\
Go2, Go1, Mini Cheetah & 0.020936 & 0.021612 & 0.015892 & 0.015490 \\
Go2, Go1, A1 & 0.018441 & 0.021740 & 0.020495 & -0.013319 \\
\bottomrule
\end{tabular}
\end{table}

% 注:表の値はevaluation_results_table.mdから取得しました

\section{エンコーダ・デコーダ分離アーキテクチャの評価}

エンコーダとデコーダを分離したアーキテクチャを用いた場合の性能を評価した.
このアーキテクチャでは,各ロボットに対して専用のデコーダを使用し,
共通のエンコーダで観測を処理する構成を採用した.

エンコーダ・デコーダ分離アーキテクチャの性能を表\ref{tab:encoder_decoder_results}に示す.

\begin{table}[t]
\centering
\caption{エンコーダ・デコーダ分離アーキテクチャの性能(1ステップあたりの平均報酬)}
\label{tab:encoder_decoder_results}
\begin{tabular}{lcc}
\toprule
\textbf{ロボット} &
\textbf{エンコーダ・デコーダ分離} &
\textbf{PPO専門家} \\
\midrule
Go2 & 0.021066 & 0.021352 \\
Go1 & - & 0.022521 \\
A1 & - & 0.020857 \\
Mini Cheetah & - & 0.021014 \\
\bottomrule
\end{tabular}
\end{table}

% 注:Go1、A1、Mini Cheetahのエンコーダ・デコーダ分離の値を実際の評価結果で埋める必要があります

実験の結果,エンコーダ・デコーダを分離したアーキテクチャでは,
全ロボットにおいて PPO 専門家ポリシーと同等の性能を達成することができた.
これは,ロボット固有の特性をデコーダで処理することで,
各ロボットに対する最適化が可能になったことを示している.
